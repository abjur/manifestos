\documentclass[]{report}
\usepackage{lmodern}
\usepackage{amssymb,amsmath}
\usepackage{ifxetex,ifluatex}
\usepackage{fixltx2e} % provides \textsubscript
\ifnum 0\ifxetex 1\fi\ifluatex 1\fi=0 % if pdftex
  \usepackage[T1]{fontenc}
  \usepackage[utf8]{inputenc}
\else % if luatex or xelatex
  \ifxetex
    \usepackage{mathspec}
  \else
    \usepackage{fontspec}
  \fi
  \defaultfontfeatures{Ligatures=TeX,Scale=MatchLowercase}
\fi
% use upquote if available, for straight quotes in verbatim environments
\IfFileExists{upquote.sty}{\usepackage{upquote}}{}
% use microtype if available
\IfFileExists{microtype.sty}{%
\usepackage[]{microtype}
\UseMicrotypeSet[protrusion]{basicmath} % disable protrusion for tt fonts
}{}
\PassOptionsToPackage{hyphens}{url} % url is loaded by hyperref
\usepackage[unicode=true]{hyperref}
\PassOptionsToPackage{usenames,dvipsnames}{color} % color is loaded by hyperref
\hypersetup{
            pdftitle={Manifestos},
            pdfauthor={Associação Brasileira de Jurimetria},
            colorlinks=true,
            linkcolor=Maroon,
            citecolor=Blue,
            urlcolor=Blue,
            breaklinks=true}
\urlstyle{same}  % don't use monospace font for urls
\usepackage{natbib}
\bibliographystyle{apalike}
\usepackage{longtable,booktabs}
% Fix footnotes in tables (requires footnote package)
\IfFileExists{footnote.sty}{\usepackage{footnote}\makesavenoteenv{long table}}{}
\IfFileExists{parskip.sty}{%
\usepackage{parskip}
}{% else
\setlength{\parindent}{0pt}
\setlength{\parskip}{6pt plus 2pt minus 1pt}
}
\setlength{\emergencystretch}{3em}  % prevent overfull lines
\providecommand{\tightlist}{%
  \setlength{\itemsep}{0pt}\setlength{\parskip}{0pt}}
\setcounter{secnumdepth}{5}
% Redefines (sub)paragraphs to behave more like sections
\ifx\paragraph\undefined\else
\let\oldparagraph\paragraph
\renewcommand{\paragraph}[1]{\oldparagraph{#1}\mbox{}}
\fi
\ifx\subparagraph\undefined\else
\let\oldsubparagraph\subparagraph
\renewcommand{\subparagraph}[1]{\oldsubparagraph{#1}\mbox{}}
\fi

% set default figure placement to htbp
\makeatletter
\def\fps@figure{htbp}
\makeatother

\usepackage[brazilian]{babel}
\usepackage[utf8]{inputenc}
\usepackage[T1]{fontenc}
\usepackage{lipsum}
\usepackage{fullwidth}
\usepackage{indentfirst}
\usepackage[left=2.5cm, right=2.5cm, top=4cm, bottom=3.8cm]{geometry}
\renewcommand{\familydefault}{\sfdefault}
\PassOptionsToPackage{dvipsnames}{xcolor}
\RequirePackage{xcolor} % [dvipsnames]
\definecolor{halfgray}{gray}{0.55} % chapter numbers will be semi transparent .5 .55 .6 .0
\definecolor{webgreen}{rgb}{0,.5,0}
\definecolor{webbrown}{rgb}{.6,0,0}
%\definecolor{Maroon}{cmyk}{0, 0.87, 0.68, 0.32}
%\definecolor{RoyalBlue}{cmyk}{1, 0.50, 0, 0}
%\definecolor{Black}{cmyk}{0, 0, 0, 0}
\usepackage{fancyhdr}
\usepackage{pdfpages}
\usepackage{amsmath}
\usepackage{graphicx}
\usepackage{listings}
\usepackage{enumitem}
\usepackage{setspace}
\usepackage{spverbatim}
\usepackage{lipsum}
\usepackage{natbib}
\usepackage{longtable}
\usepackage{booktabs}
\usepackage{background}

\newcommand{\prestadorEmpresaFoot}{Associação Brasileira de Jurimetria}
\newcommand{\prestadorEmpresa}{Associação Brasileira de Jurimetria}
\newcommand{\prestadorRepr}{Marcelo Guedes Nunes}
\newcommand{\prestadorEnderecoFoot}{Rua Gomes de Carvalho, 1356, 2º andar. CEP 04547-005 - São Paulo, SP, Brasil.}
\newcommand{\prestadorEndereco}{Rua Gomes de Carvalho, 1356, 2º andar}
\newcommand{\prestadorEnderecoComp}{CEP 04547-005 - São Paulo, SP, Brasil}
\newcommand{\prestadorSite}{\url{http://abj.org.br}}
\newcommand{\prestadorEmail}{contato@abj.org.br}
% \includegraphics[trim=left bottom right top, clip]{file}
\newcommand{\logo}{\includegraphics[width=0.21\textwidth, trim=0cm 0cm 0cm 11.1cm, clip]{imgs/logo_abj.png}}

\newcommand{\tipoTrabalho}{}
\newcommand{\numero}{XXX}

\newcommand{\clienteEmpresa}{XXX}
\newcommand{\clienteEndereco}{XXX}
\newcommand{\clienteEnderecoComp}{XXX}
\newcommand{\clienteRepr}{XXX}
\newcommand{\clienteEmail}{XXX}

% \fancypagestyle{firststyle}
% {
%     \pagestyle{fancy}
%     \lhead{\thepage}
%     \chead{}
%     \rhead{\logo{}}
%     \cfoot{
%         \footnotesize{\prestadorEmpresaFoot{}} \\
%         \footnotesize{\prestadorEnderecoFoot{}} \\
%         \footnotesize{\prestadorSite{}}
%     }
%     \renewcommand{\headrulewidth}{0.5pt}
%     \renewcommand{\footrulewidth}{0.5pt}
%     \setlength{\headsep}{.5in}
% }
%
% \pagestyle{firststyle}

\setlength{\parindent}{2em}
% \setlength{\parskip}{1em}
% \renewcommand{\baselinestretch}{1.2}
% % \usepackage{palatino}
% \renewcommand{\familydefault}{\sfdefault} % sans serif
% \fontfamily{ppl}\selectfont

%\renewcommand{\baselinestretch}{1.4}

\backgroundsetup{
scale=1,
angle=0,
opacity=1,
color=black,
contents={\begin{tikzpicture}[remember picture,overlay]
\node at ([xshift=-4.35cm,yshift=-2.5cm] current page.north east) % Adjust the position of the logo.
{\logo}; % logo goes here
\end{tikzpicture}}
}

\usepackage{float}
\let\origfigure\figure
\let\endorigfigure\endfigure
\renewenvironment{figure}[1][2] {
    \expandafter\origfigure\expandafter[H]
} {
    \endorigfigure
}

\title{Manifestos}
\author{Associação Brasileira de Jurimetria}
\date{09 de novembro de 2017}

\begin{document}
\maketitle

\chapter*{Manifestos}\label{manifestos}


Esse documento contém os manifestos da ABJ.

\chapter{Manifesto por dados abertos}\label{manifesto-por-dados-abertos}

\section{Introdução (motivação)}\label{introducao-motivacao}

Não se muda aquilo que se ignora

É estratégico (tribunais e hospitais)

É grande

Os dados são públicos

Mesmo assim, temos problemas de privacidade

Queremos usar para fazer pesquisa, para adm de tribunais e políticas
públicas.

\section{Desafios}\label{desafios}

Existem inúmeras ferramentas públicas e privadas para busca e
recuperação de processos. Os sistemas são eficazes, mas são todos
voltados para a busca de informações individuais. Se uma pessoa tiver o
número identificador, ela achará informações do processo. Se precisar
uma lista de processos, poderá utilizar ferramentas de busca.

Cientistas de dados, no entanto, precisam ter a possibilidade de
exportar os dados completos ou algum recorte da população para
planilhas. Existem muitos exemplos de páginas úteis para cientistas de
dados, como IpeaData, Datasus, IBGE, entre outros.

Também é possível utilizar \emph{Application Programming Interfaces}
(APIs) para obter dados de \emph{tweets}, publicações no Facebook, entre
outros. O importante é notar que os sistemas voltados para análise de
dados são em sua maioria voltados para extração de informações de muitos
indivíduos. Os dados são organizados para análise e não para consulta
individual. Muitas vezes é necessário limpar a base, mas isso faz parte
do fluxo da ciência de dados \citep{wickham2016r}.

Na área do direito, o pesquisador fica numa situação complicada, pois
precisa de dados da população ou de uma amostra, com linhas e colunas,
numa planilha padronizada. No entanto, tudo o que consegue encontrar são
documentos individuais, listagens de processos e arquivos em formato
fechado, como o \emph{Portable Document Format} (PDF).

Muitas vezes os dados estão disponíveis em páginas web mas é muito
demorado buscar todos os casos que precisamos manualmente ou através de
ofícios. Por isso, é usual construir \emph{web scrapers}, que são robôs
que baixam as páginas automaticamente e depois as tranformam em dados
estruturados.

Atualmente, a utilização \emph{web scrapers} é indispensável em estudo
jurimétricos. As pesquisas realizadas pela ABJ foram fortemente
influenciadas por essas ferramentas.

Contudo, são raros os profissionais que dominam esse conhecimento. A ABJ
disponibiliza abertamente todo seu aparato técnico \footnote{Disponíveis
  nos links:\url{https://github.com/abjur} (códigos genéricos da ABJ),
  \url{https://github.com/courtsbr} (web scrapers), e
  \url{https://github.com/decryptr} (ferramentas para quebrar CAPTCHAs).},
mas as ferramentas não são capazes de resolver todos os problemas. Além
disso, os sistemas dos Tribunais colocam impedimentos técnicos de
acesso, dificultando a execução de pesquisas que poderiam ser benéficas
para os próprios Tribunais.

Há muitos exemplos em que simplesmente não é possível obter as
informações que necessitamos. Em muitos casos a única forma de acessar
os dados é a partir da Lei de Acesso à Informação. Apesar da LAI ser um
grande avanço, utilizá-la para todas as demandas é ineficiente pois
congestiona os setores administrativos e técnicos dos Tribunais.

A solução mais eficaz para o problema de acesso aos dados envolve
modificar os sites dos Tribunais, permitindo extrações de dados e
disponibilizar APIs que permitam os pesquisadores de buscar as
informações públicas de maneira segura e organizada.

Essas ferramentas são simples de construir para entidades como os
Tribunais, que geralmente têm equipes de Tecnologia da Informação de
altíssima qualidade. A solução não causaria impactos negativos nos
sistemas; pelo contrário: ao permitir que os dados sejam baixados de
forma programática, é possível controlar o volume de dados transferido
por unidade de tempo, evitando que os servidores fiquem sobrecarregados.

\section{Desafio}\label{desafio}

Em teoria, a solução da API é perfeita. O problema é que falta uma
padronização concreta para que

Desafio: como conectar sistemas, sem necessariamente modificar a
estrutura dos tribunais?

O erro está em querer alinhamento de sistemas, e não alinhamento focado
em pesquisa.

As pesquisas readequam o sistema

\section{Solução}\label{solucao}

Partir do processo de realização de pesquisas. Ciclo da ciência de
dados?

A pesquisa no direito com fins de elaboração de políticas públicas ou
administração de tribunais seguem padrões específicos. Na história da
ABJ, executamos pesquisas em praticamente todas as esferas do direito.
Já pesquisamos sobre direito tributário, processos de adoção, direito do
consumidor, processos de homicício, falências e recuperações, entre
muitos outros. Com base nessa experiência e com base no que observamos
na Society of Empirical Legal Studies, montamos uma lista do que
acreditamos ser os objetos de pesquisa mais comuns no direito.

\subsection{Contagens e valores por
tema/assunto}\label{contagens-e-valores-por-temaassunto}

Trata-se do tema mais comum em pesquisa do direito. Queremos contar
processos por tipo, por valor, por tribunal etc. Por exemplo, Justiça em
Números.

\subsection{Tempos e complexidade
processual}\label{tempos-e-complexidade-processual}

\subsection{Resultados e reformas dos processos
judiciais}\label{resultados-e-reformas-dos-processos-judiciais}

Para entender o que é o direito, precisamos entender o que os juízes
decidem. Queremos saber se as decisões variam muito dentro de um mesmo
tema, se existe discriminação e se é possível

\subsection{análise de impacto
regulatório}\label{analise-de-impacto-regulatorio}

O que essas pesquisas tem em comum?

Estrutura de dados: infos básicas, partes, movimentações, decisões,
documentos.

\section{Propostas}\label{propostas}

Não queremos que os tribunais mandem, queremos poder extrair dos
tribunais

Proposta de como fazer: API pública, camada de APIs

Passo 1: Disponibilização de API pública de infosbasicas, partes e
movimentacoes

Dados do Selo Justiça em números!

Passo 2: Diário oficial digital no CKAN - arrumar distribuicao

Passo 3: input/output: O que foi pedido, o que foi concedido?

Passo 4: Ligando processos, partes, advogados, teses e magistrados

Passo 5: Integração com polícia, defensoria e ministério público, entre
outras (para segpub)

\section{Aplicações}\label{aplicacoes}

Parágrafo introdutório

\begin{itemize}
\tightlist
\item
  Para o judiciário (assuntos processuais e padronização)
\item
  Para o cidadão (acesso à justiça)
\item
  Para empresas (acompanhamento processual)
\item
  Para legaltechs (acelerar fluxos de mediação)
\item
  Para magistrados (tomada de decisão, conexão com outros sistemas)
\item
  Para advogados (acompanhamento processual e jurisprudência)
\end{itemize}

Economia de bilhões. Como? Criar uma conta

\section{Próximos passos}\label{proximos-passos}

A solução precisa ser comandada pelo DPJ-CNJ

Juntar com Selo Justiça em Números

Os tribunais já recebem diversas propostas. É preciso alinhar e limpar

\begin{enumerate}
\def\labelenumi{\arabic{enumi}.}
\tightlist
\item
  Coordenação: DPJ
\item
  equipe técnica

  \begin{itemize}
  \tightlist
  \item
    pesquisadores da ABJ e do DPJ
  \item
    arquitetos de TI dos sistemas atuais (saj, pje)
  \end{itemize}
\item
  equipe articuladora

  \begin{itemize}
  \tightlist
  \item
    representantes da OAB, GEAL-CRASP, conselheiros do CNJ
  \item
    Representantes dos tribunais superiores
  \item
    universidades: Insper, GV, PUC, USP
  \end{itemize}
\end{enumerate}

Parágrafo de finalização

\bibliography{bibliography/book.bib,bibliography/packages.bib}

\end{document}
